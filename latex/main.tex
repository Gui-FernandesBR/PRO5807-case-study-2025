\documentclass[12pt,a4paper]{article}

% Language and encoding
\usepackage[utf8]{inputenc}
\usepackage[english]{babel}

% comment
\usepackage{comment}

% Page layout
\usepackage[a4paper,margin=1in]{geometry}
\usepackage{setspace}
\onehalfspacing

% Fonts and formatting
\usepackage{lmodern}
\usepackage[T1]{fontenc}
\usepackage{microtype}

% Graphics and tables
\usepackage{graphicx}
\usepackage{booktabs}
\usepackage{array}
\usepackage{caption} % Added for captions
\usepackage{float}   % Improved float handling
\usepackage{makecell}

% Math and symbols
\usepackage{amsmath}
\usepackage{amssymb}
\usepackage{siunitx}

% Hyperlinks
\usepackage{hyperref}
\hypersetup{
    colorlinks=true,
    linkcolor=blue,
    filecolor=magenta,
    urlcolor=blue,
    pdftitle={Usemore Soap Company Case Study},
    pdfauthor={},
    pdfsubject={},
    pdfkeywords={}
}

% Custom commands
\newcommand{\cwt}{\text{cwt}}
\newcommand{\dollar}{\$}

\captionsetup{labelfont=bf, font=small, labelsep=colon} % Optional: caption style

% Title and author
\title{Usemore Soap Company: A Warehouse Location Case Study}
\author{}
\date{\today}

\begin{document}
\maketitle


\section{Enunciado}

The Usemore Soap Company produces a line of cleaning compounds, used mainly for industrial and institutional purposes.
Typical products include general cleaning compounds, dishwasher powders, rinse agents, hand soaps, motor vehicle washing compounds, and cleaning products for the food industry.
The product line is composed of more than 200 products and nearly 800 individual product items.
Package sizes range from 18-pound cases to large metal drums weighing 550 pounds.

Sales are generated throughout the 48 contiguous United States, with additional sales in Hawaii, Alaska, and Puerto Rico.
Customers typically purchase in quantities less than 10,000 pounds, that is, less-than-truckload (LTL) quantities.
A few customers purchase in truckload and bulk quantities.
Annual LTL sales, which pass through the warehouses, are running at the 150 million pound level.
Volume sales, which are served directly from plants, add another 75 million pounds.
These sales represent approximately \$160 million in revenue.

The primary marketing effort comes from a direct selling force operating under the incentive of a liberal sales commission structure.
Salespeople look upon themselves as individual entrepreneurs and have a great deal of autonomy within the company.
This marketing strategy has generally proved successful for the company, as the company has often been referred to as one of the most profitable divisions within its widely
diversified parent organization.

In spite of the high profitability, company management is concerned about the costs of producing and distributing the product line to maintain its competitive edge.
Growth and shifting demand patterns are straining the production capacity of the four existing plants.
In addition, changing costs of distribution, as well as the fact that the distribution network has not been studied in 12 years, raise questions about the proper placement of the warehouses.
What follows is a summary of the problem conditions being faced by management.
You are to suggest an improved distribution network that meets the stated customer service policy and minimizes total network production-distribution costs.

\subsection{BACKGROUND}

The current distribution network consists of four product line plants located
at Covington, Kentucky, New York, New York, Arlington, Texas, and Long Beach, California.
The plants are currently producing product for their low-volume customers at the level of 595,102 cwt., 390,876 cwt., 249, 662 cwt., and 241, 386 cwt., respectively.
This output is shipped from plants either to field warehouses in the distribution network or to customers within the local areas of the plants.
In the latter case, plants serve as field warehouses as well as producing centers.

Warehousing takes place at 18 public warehouses and at the four plant locations, as shown in Table \ref{tab:tabela1-enunciado}.
These warehouses are dispersed in such a fashion that the majority of the customers are within a one-day delivery time frame of a stocking point; that is, approximately 300 miles.
Except for the plants serving as warehouses, the warehouses are supplied in full truckload quantities.
Less-than-truckload shipments serve customers.
Customer order processing takes place at each warehouse location.
In addition, two potential plant sites are being considered at Chicago, Illinois, and Memphis, Tennessee.

\begin{table}[!h]
    \caption{Current Plant and Public Warehouse Locations}
    \centering
    \label{tab:tabela1-enunciado}
    \begin{tabular}{|c|c|c|}
    \hline
    No & Location (city) & Location (state) \\
    \hline
    1  & Covington       & KY               \\
    2  & New York        & NY               \\
    3  & Arlington       & TX               \\
    4  & Long Beach      & CA               \\
    5  & Atlanta         & GA               \\
    6  & Boston          & MA               \\
    7  & Buffalo         & NY               \\
    8  & Chicago         & IL               \\
    9  & Cleveland       & OH               \\
    10 & Devenport       & IA               \\
    11 & Detroit         & MI               \\
    12 & Grand Rapids    & MI               \\
    13 & Greensboro      & NC               \\
    14 & Kansas City     & KS               \\
    15 & Baltimore       & MD               \\
    16 & Memphis         & TN               \\
    17 & Milwaukee       & WI               \\
    18 & Orlando         & SL               \\
    19 & Pittsburgh      & PA               \\
    20 & Portland        & OR               \\
    21 & W Sacramento    & CA               \\
    22 & W Chester       & PA               \\
    \hline
\end{tabular}
\end{table}



Additional warehouse sites are considered at the locations shown in Table \ref{tab:tabela2-enunciado}.

\begin{table}[!h]
    \centering
    \caption{Possible Public Warehouse Locations}
    \label{tab:tabela2-enunciado}
    \begin{tabular}{|l|l|l|}
    \hline
    No & Location (city) & Location (state) \\
    \hline
    23 & Albuquerque     & NM               \\
    24 & Biling          & MT               \\
    25 & Denver          & CO               \\
    26 & El Paso         & TX               \\
    27 & Camp Hill       & PA               \\
    28 & Houston         & TX               \\
    29 & Las Vegas       & NV               \\
    30 & Minneapolis     & MN               \\
    31 & New Orleans     & LA               \\
    32 & Phoenix         & AZ               \\
    33 & Richmond        & VA               \\
    34 & St Louis        & MO               \\
    35 & Salt Lake City  & UT               \\
    36 & San Antonio     & TX               \\
    37 & Seattle         & WA               \\
    38 & Spokane         & WA               \\
    39 & San Francisco   & CA               \\
    40 & Indianapolis    & IN               \\
    41 & Louisville      & KY               \\
    42 & Columbus        & OH               \\
    43 & New York        & NY               \\
    44 & Hartford        & CT               \\
    45 & Miami           & FL               \\
    46 & Mobile          & AL               \\
    47 & Memphis *       & TN               \\
    48 & Chicago *       & IL               \\
    \hline
\end{tabular}
\end{table}


Potential warehouse sites are made based on sales personnel's suggestions, favorable warehousing rates, good warehousing service availability, proximity to demand concentrations, and filling out of the distribution network.
Of the existing and potential warehouse sites, it is hoped that an improved mix of warehouses can be found.
In addition, plant expansion, either at existing sites or at new sites, will be needed to meet future demand projections.
Specifically, answers to the following questions are sought:

\begin{enumerate}
    \begin{singlespace}
        \item How many warehouses should be operated now and in the future?
        \item Where should they be located?
        \item Which customers and associated demand should be assigned to each warehouse and plant?
        \item Which warehouses should be supplied from each plant?
        \item Should production capacity be expanded? When, where, and by how much?
        \item What level of customer service should be provided?
    \end{singlespace}
\end{enumerate}

\subsection{SALES DATA}

Manufacturing of soap liquids and powders is an uncomplicated duplicated process, which contributes to substantial competition in the marketplace.
The undifferentiated nature of soap products results in keen competition in both price and service.
Customer service is of particular concern because it is directly affected by the choice of warehouses.
No specific dollar figure can be placed on the total value of good distribution service, as it depends on customer attitudes about service and resulting patronage.
The general feeling in the company is that service should be maintained at a high level so as not to jeopardize sales.
A ``high'' level of service is taken to mean delivery time of 24 to 48 hours or less.
This generally places customers somewhere between 300 and 600 miles of warehouses.
Annual sales for the products that move through the warehousing network are 147 million pounds for annual revenue of slightly more than \$100 million.
Sales are distributed similarly to population centers with an average profit margin of 20 percent.
Figure \ref{fig:salesTerritories} shows the six major sales territories, with sales volume in pounds by state.

\begin{figure}
    \centering
    \includegraphics[width=0.8\textwidth]{images/enunciado/figura01.png}
    \caption{Usemore Soap Company Annual Sales in Cwt. by State, with Major Sales Districts Defined}\label{fig:salesTerritories}
\end{figure}


The company has more than 70,000 individual customer accounts, and these are aggregated into 191 active demand centers.
A demand center is a grouping of zip code areas into a zip sectional center as the focus of the collected demand.
These demand centers, along with how they are currently being served, are given in Table \ref{tab:tab3-enunciado}.
In addition, the sales territory in which the demand center is grouped is shown.
The five-year plan shows volume growth throughout the United States.
This growth will not be uniform due to population and business migration patterns, competition, and varying promotional efforts.
The changes in volume compared with current volume levels are projected by sales territory as follows (Table \ref{tab:tabelaa-enunciado}):

\begin{table}[!h]
    \centering
    \caption{Sales Territory Growth Factors} \label{tab:tabelaa-enunciado}
    \begin{tabular}{|c|c|c|}
    \hline
    \textbf{Region No.} & \textbf{Sales Territory} & \textbf{5-Year Growth Factor} \\
    \hline
    1                   & Northeast                & 1,3                           \\
    2                   & Southeast                & 1,45                          \\
    3                   & Midwest                  & 1,25                          \\
    4                   & Northwest                & 1,2                           \\
    5                   & Southwest                & 1,15                          \\
    6                   & West                     & 1,35     \\
    \hline
\end{tabular}
\end{table}

\subsection{PRODUCTION COSTS AND CAPACITIES}

The production variable costs at existing plants vary by location.
This variance results from labor rate differences, volume purchases of raw materials, and inbound transport cost differences due to the proximity of the plants to major raw materials sources.
These costs are listed next (Table \ref{tab:tableb-enunciado}).

\begin{table}[!h]
    \centering
    \caption{Production Costs at Existing Plants}
    \label{tab:tableb-enunciado}
    \begin{tabular}{|c|c|}
    \hline
    \textbf{Plant} & \textbf{Variable production cost} \\
    Covington, KY  & 21,0                              \\
    New York, NY   & 19,9                              \\
    Arlington, TX  & 21,6                              \\
    Long Beach, CA & 21,1                              \\
    \hline
\end{tabular}
\end{table}

The potential plant at Chicago has an estimated cost of \$21.0 per cwt.;
and the Memphis plant has a cost of \$20.6 per cwt.
Expansion at any existing plant site would have the current variable cost.
Fixed costs are not included for existing plants because these are sunk costs.
However, to construct a new plant or expand an existing one would cost a minimum of \$4 million.
This cost would result in an output for the plant (or an output increase in the case of a plant addition) of up to 1 million cwt. per year for the near future.

According to current distribution patterns, the existing plants are producing, relative to throughput capacity (in cwt.), at the following rates (Table \ref{tab:tablec-enunciado}):

\begin{table}[!h]
    \centering
    \caption{Current Production Rates} \label{tab:tablec-enunciado}
    \begin{tabular}{|l|l|l|l|}
        \hline
        \makecell{\textbf{Plant}} &
        \makecell{\textbf{Current capacity} \\ \textbf{(cwt.)}} &
        \makecell{\textbf{Current Production} \\ \textbf{(cwt.)}} &
        \makecell{\textbf{Percent of} \\ \textbf{Capacity}} \\
        \hline
        Covington, KY  & 620000                           & 595102                             & 96\%                         \\
        New York, NY   & 430000                           & 390876                             & 91\%                         \\
        Arlington, TX  & 300000                           & 249662                             & 83\%                         \\
        Long Beach, CA & 280000                           & 241386                             & 86\%                         \\
        \textbf{Total} & \textbf{1630000}                 & \textbf{1477026}                   & \textbf{91\%}                \\
        \hline
    \end{tabular}
\end{table}

\subsection{WAREHOUSING RATES AND CAPACITIES}

Company contracts with public warehousemen show that rates are categorized as storage, handling, and acessorial.
Storage rates are quoted on a \$/cwt./month basis of average inventory held.
Handling charges are incurred whenever in or out movement of the product occurs and are assessed on a \$/cwt. basis.
Acessorial charges are for a number of services, such as bill of lading preparation, local delivery, and stock status reporting.
Similar charges are estimated for the four plant warehouses as a fair share of production operations.

Also associated with warehousing are the stock replenishment costs.
These are costs for preparing the paperwork for normal replenishment and the expediting of stock into the warehouse.
Stock order costs as well as customer order costs are computed by multiplying the average cost per order by the average number of orders for the warehouse.

The warehouse-related costs and other associated information are given in Table 3.
Costs for existing points are taken from company records.
Those for potential warehouses are determined from quotas by warehousemen in the appropriate cities.
Estimates are made of costs where such information is not otherwise available.

There are no effective capacity limits on public warehousing.
Usemore's space need is a small fraction of a public warehouseman's total capacity.
On the other hand, a throughput of at least 10,400 cwt. per year, or a replenishment truckload every two weeks, is the desired minimum throughput needed to open a warehouse.
Available space is limited at the four current plant sites.
The stocking limits in terms of throughput at Covington = 450,000 cwt., New York = 380,000 cwt., Arlington = 140,000 cwt., and Long Beach = 180,000 cwt.

\subsection{TRANSPORTATION COSTS}

Three general transportation cost types are important to Usemore: inbound, outbound, and local delivery transport charges.
Inbound transportation costs to a warehouse depend on the volume shipped and the distance between plant and warehouse.
A sampling of truck common carrier rates at various distances from the plants for full truckload shipments shows that the transport rate between a plant and warehouse (P-W) can be reasonably approximated by a linear function.
That is, the truckload rate is

\begin{equation}
    \text{P-W rate (\$/cwt.)} = 0.92 + 0.0034 \cdot d \quad \text{(miles)}
\end{equation}

where $d$ is the distance between two points (For simplicity, one aggregated relationship is shown.
In practice, several such relationships would be used to reflect the rate difference caused by geographic locations of the shipment origin points).
Total inbound transport costs are determined by multiplying the P-W rate by the volume assigned to flow between the plant and warehouse.

Warehouse outbound transport costs depend on the distance that a customer is from the warehouse.
If the customer is roughly within 30 miles of the warehouse, local cartage rates generally apply.
These local delivery rates are shown by warehouse in Table \ref{tab:tab3-enunciado}
For distances greater than 30 miles, a linear function similar to that for the inbound rates can be developed.
Given the average shipment size from the warehouses of approximately 1,000 pounds, the
warehouse to customer (W-C) rate function is

\begin{equation}
    \text{W-C rate (\$/cwt.)} = 5.45 + 0.0037 \cdot d \quad \text{(miles)}
\end{equation}

Computation of total warehouse outbound transport costs is carried out in the same manner as for inbound transport costs.

\begin{table}[]
    \centering
    \caption{Stocking Point Rate and Order Size Information}\label{tab:tab3-enunciado}
    \resizebox{\textwidth}{!}{%
    \begin{tabular}{|c|c|c|c|c|c|c|c|}
    \hline
    \makecell{\textbf{Warehouse} \\ \textbf{No.}} &
    \makecell{\textbf{Storage} \\ (\$/cwt)} &
    \makecell{\textbf{Handling} \\ (\$/cwt)} &
    \makecell{\textbf{Stock Order} \\ \textbf{Processing} \\ (\$/order)} &
    \makecell{\textbf{Stock Order} \\ \textbf{Size} \\ (cwt/order)} &
    \makecell{\textbf{Customer Order} \\ \textbf{Processing} \\ (\$/order)} &
    \makecell{\textbf{Customer Order} \\ \textbf{Size} \\ (cwt/order)} &
    \makecell{\textbf{Local delivery} \\ \textbf{rate} \\ (\$/cwt)} \\
    \hline
    1                        & 0,0672                 & 0,46                       & 18                                         & 400                       & 1,79                                            & 9,05                                     & 1,9                                   \\
    2                        & 0,0567                 & 0,54                       & 18                                         & 400                       & 1,74                                            & 10,92                                    & 3,89                                  \\
    3                        & 0,0755                 & 0,38                       & 18                                         & 400                       & 2,71                                            & 11,59                                    & 2,02                                  \\
    4                        & 0,0735                 & 0,59                       & 18                                         & 400                       & 1,74                                            & 11,3                                     & 4,31                                  \\
    \textbf{5}               & \textbf{0,0946}        & \textbf{0,5}               & \textbf{18}                                & 401                       & 0,83                                            & 9,31                                     & 1,89                                  \\
    6                        & 0,1802                 & 0,75                       & 18                                         & 405                       & 3,21                                            & 9                                        & 4,7                                   \\
    7                        & 0,0946                 & 0,74                       & 18                                         & 405                       & 1,23                                            & 8,37                                     & 1,55                                  \\
    8                        & 0,2072                 & 1,14                       & 18                                         & 405                       & 1,83                                            & 13,46                                    & 1,79                                  \\
    9                        & 0,1802                 & 1,62                       & 18                                         & 409                       & 4,83                                            & 9,69                                     & 4,92                                  \\
    10                       & 0,1442                 & 1,14                       & 18                                         & 410                       & 2,74                                            & 8,28                                     & 2,23                                  \\
    11                       & 0,0946                 & 1,04                       & 18                                         & 409                       & 3,93                                            & 10,2                                     & 1,81                                  \\
    12                       & 0,1982                 & 1,06                       & 18                                         & 410                       & 3,18                                            & 15                                       & 1                                     \\
    13                       & 0,0766                 & 1,06                       & 18                                         & 400                       & 1,08                                            & 9,07                                     & 1,63                                  \\
    14                       & 0,1262                 & 1,22                       & 18                                         & 423                       & 1,56                                            & 11,72                                    & 1,17                                  \\
    15                       & 0,1126                 & 0,82                       & 18                                         & 426                       & 1,2                                             & 9,35                                     & 1,73                                  \\
    16                       & 0,991                  & 0,64                       & 18                                         & 433                       & 1,78                                            & 8,7                                      & 0,5                                   \\
    17                       & 0,1577                 & 0,71                       & 18                                         & 394                       & 5,33                                            & 8,07                                     & 1,46                                  \\
    18                       & 0,1307                 & 0,79                       & 18                                         & 398                       & 0,91                                            & 7,66                                     & 2,29                                  \\
    19                       & 0,1487                 & 1,15                       & 18                                         & 399                       & 2,08                                            & 9,39                                     & 2,2                                   \\
    20                       & 0,2253                 & 0,8                        & 18                                         & 490                       & 1,1                                             & 7,31                                     & 1,49                                  \\
    21                       & 0,137                  & 1,39                       & 18                                         & 655                       & 1,7                                             & 9,31                                     & 2,72                                  \\
    22                       & 0,0991                 & 0,83                       & 18                                         & 400                       & 2,46                                            & 10,14                                    & 4,17                                  \\
    23                       & 0,126                  & 0,59                       & 18                                         & 110                       & 2,33                                            & 5,07                                     & 2,37                                  \\
    24                       & 0,0637                 & 0,45                       & 18                                         & 134                       & 1,88                                            & 6,8                                      & 1,36                                  \\
    25                       & 0,0946                 & 1,68                       & 18                                         & 341                       & 2,58                                            & 6,83                                     & 2,21                                  \\
    26                       & 0,1216                 & 0,88                       & 18                                         & 149                       & 1,83                                            & 14,32                                    & 0,8                                   \\
    27                       & 0,0721                 & 0,55                       & 18                                         & 198                       & 1,83                                            & 7,38                                     & 3,88                                  \\
    28                       & 0,1532                 & 0,8                        & 18                                         & 420                       & 1,58                                            & 9,7                                      & 2,14                                  \\
    29                       & 0,1172                 & 1,04                       & 18                                         & 287                       & 0,78                                            & 7,52                                     & 1,51                                  \\
    30                       & 0,108                  & 1,46                       & 18                                         & 408                       & 5,33                                            & 11,46                                    & 1,7                                   \\
    31                       & 0,1487                 & 0,95                       & 18                                         & 340                       & 1,36                                            & 10,48                                    & 1,63                                  \\
    32                       & 0,1352                 & 0,69                       & 18                                         & 333                       & 1,5                                             & 6,67                                     & 1,66                                  \\
    33                       & 0,1126                 & 0,64                       & 18                                         & 277                       & 2,33                                            & 11,98                                    & 1,54                                  \\
    34                       & 0,1712                 & 1,35                       & 18                                         & 398                       & 0,93                                            & 10,13                                    & 1,84                                  \\
    35                       & 0,1261                 & 0,79                       & 18                                         & 434                       & 2,08                                            & 6,81                                     & 1,58                                  \\
    36                       & 0,1352                 & 0,8                        & 18                                         & 323                       & 0,88                                            & 7,67                                     & 1,93                                  \\
    37                       & 0,2704                 & 0,96                       & 18                                         & 423                       & 0,89                                            & 8,57                                     & 3,08                                  \\
    38                       & 0,225                  & 0,8                        & 18                                         & 425                       & 2,88                                            & 7,61                                     & 1,43                                  \\
    39                       & 0,1487                 & 4,49                       & 18                                         & 400                       & 1,46                                            & 7,55                                     & 6,44                                  \\
    40                       & 0,2073                 & 1,14                       & 18                                         & 400                       & 2,75                                            & 10,13                                    & 2,83                                  \\
    41                       & 0,2073                 & 1,14                       & 18                                         & 400                       & 2,75                                            & 10,13                                    & 2,83                                  \\
    42                       & 0,1802                 & 1,62                       & 18                                         & 400                       & 2,75                                            & 10,13                                    & 4,81                                  \\
    43                       & 0,2613                 & 1,39                       & 18                                         & 400                       & 2,71                                            & 11,59                                    & 3,89                                  \\
    44                       & 0,1396                 & 0,71                       & 18                                         & 400                       & 2,04                                            & 9,37                                     & 3,89                                  \\
    45                       & 0,1036                 & 0,55                       & 18                                         & 400                       & 2,75                                            & 10,13                                    & 1,74                                  \\
    46                       & 0,0946                 & 0,55                       & 18                                         & 400                       & 1,74                                            & 9,31                                     & 1,89                                  \\
    47                       & 0,0682                 & 0,64                       & 18                                         & 400                       & 1,78                                            & 8,7                                      & 0,5                                   \\
    48                       & 0,0682                 & 1,22                       & 18                                         & 400                       & 1,79                                            & 9,05                                     & 1,55 \\
    \hline
\end{tabular}%
}
\end{table}

\subsection{INVENTORY COSTS}

Inventory costs depend on the average inventory maintained at a warehouse and the inventory rate factors that apply to the inventory level.
These rate factors include the cost of capital, personal property taxes, and insurance costs.
The average inventory at a warehouse will vary by the demand on the warehouse and by the method used to control the inventory.
A mathematical function to express inventory based on annual warehouse throughput is found by plotting the annual average inventory against annual throughput at each active stocking point.
The resulting curve is shown in Figure \ref{fig:figure2Inventory}.

\begin{figure}[!h]
    \centering
    \includegraphics[width=0.8\textwidth]{images/enunciado/figura02.png}
    \caption{The Inventory-to-Warehouse Throughput Relationship for the Usemore Soap Company}\label{fig:figure2Inventory}
\end{figure}

Knowing that the annual cost-to-carry-inventory rate is approximately 12 percent of the average product value of \$26 per cwt., the total cost to carry inventory at each warehouse is given by

\begin{equation}
    IC_{i} = (0.12)(26)(11.3 d_{i}^{0.58}) = 35.3 D_{i}^{0.58}
\end{equation}

where

\begin{align*}
    IC_{i} &= \text{inventory carrying cost at warehouse i (\$)} \\
    D_{i}  &= \text{annual throughput at warehouse i (cwt)}
\end{align*}

\subsection{WAREHOUSE OPERATING COSTS}

Warehouse operating costs refer to the combination of storage and handling costs incurred resulting from assigning demand to warehouses.
Storage costs are computed by taking the storage rate and multiplying it by an estimate of the average inventory in the warehouse.
Mathematically, this can be expressed as

\begin{equation}
    SC_{i} = SR_{i} \cdot (26) \cdot (11.3 D_{i}^{0.58})
\end{equation}

where

\begin{align*}
    SC_{i} &= \text{annual cost of stock at warehouse i (\$)} \\
    SR_{i}  &= \text{storage rate from warehouse i from Table 4} \\
    D_{i}   &= \text{annual demand throughput at warehouse i (cwt)}
\end{align*}

Handling costs are strictly a function of the warehouse throughput.
They are determined by  the handling rate multiplied by the throughput, or

\begin{equation}
    HC_{i} = HR_{i} \cdot D_{i}
\end{equation}

where

\begin{align*}
    HC_{i}  &= \text{annual handling cost at warehouse i (\$)} \\
    HR_{i}  &= \text{handling rate at warehouse i from Table 4}
\end{align*}


\subsection{ORDER-PROCESSING COSTS}

Order-processing costs refer to the charges incurred in handling the paperwork associated with stock replenishment and customer orders.
Both types of costs are computed for each warehouse in essentially the same way.
That is, the order-processing rate is multiplied by the annual demand on the warehouse and the result divided by the order size.


\subsection{TOTAL COSTS}

The total costs for various production distribution configurations can be determined by summing all the relevant costs.
For the Usemore Soap Company, these are production costs; warehouse
operating costs (storage, handling, stock order processing, and customer order processing);
transportation costs (warehouse inbound, outbound, and local delivery); and inventory-carrying costs.
Changing the number and location of plants and warehouses will cause a
change in the balance of these cost factors.
For example, adding warehouses will typically reduce transportation costs but increase inventory costs, as well as affect customer service.
Assessing the trade-offs between costs and customer service is at the heart of this problem type.
The cost and customer service summaries for the current network design are shown in Tables \ref{tab:table4-enunciado} and \ref{tab:table5-enunciado}.
At present, Usemore Soap is able to place 93 percent of its demand within 300 miles of warehouses for a total annual cost of
\$42,112,463.


\begin{table}[!h]
    \caption{Benchmark Customer Service Profile}
    \centering
    \label{tab:table4-enunciado}
    \begin{tabular}{|c|c|c|c|}
    \hline
    \makecell{\textbf{Warehouse to} \\ \textbf{customer distance}} &
    \makecell{\textbf{Percent} \\ \textbf{of demand}} &
    \makecell{\textbf{Cumulative} \\ \textbf{percent of demand}} &
    \makecell{\textbf{Total demand} \\ \textbf{(cwt.)}} \\
    \hline
    0-100 mi.    & 56,40\% & 56,40\% & 833,043 \\
    101-200      & 21,30\% & 77,70\% & 314,607 \\
    201-300      & 15,70\% & 93,40\% & 231,893 \\
    301-400      & 2,10\%  & 95,50\% & 31,018  \\
    401-500      & 1,50\%  & 97\%    & 22,155  \\
    501-600      & 0,50\%  & 97,50\% & 7,385   \\
    601-700      & 2,00\%  & 99,50\% & 29,541  \\
    701-800      & 0,50\%  & 100\%   & 7,384   \\
    801-900      & 0\%     & 100\%   & 0       \\
    901-1000     & 0\%     & 100\%   & 0       \\
    $>$1000       & 0\%     & 100\%   & 0       \\
                  & 100\%   &         & 1,477,026 \\
    \hline
    \end{tabular}
\end{table}

\begin{table}[!h]
    \caption{Cost profile for the Current Distribution Network}
    \label{tab:table5-enunciado}
    \centering
    \begin{tabular}{|l|r|}
    \hline
    \textbf{Cost Category}                   & \textbf{Cost} \\
    \hline
    Production                               & \$ 30.761.520 \\
    Warehouse operations                     & \$  1.578.379 \\
    Order processing                         & \$    369.027 \\
    Inventory carrying                       & \$    457.290 \\
    Transportation - Inbound to warehouse    & \$  2.050.367 \\
    Transportation - Outbound from warehouse & \$  6.895.880 \\
    Total cost                               & \$ 42.112.463 \\
    \hline
\end{tabular}
\end{table}


\subsection{A COMPUTER-ASSISTED ANALYSIS}

Although enough data have been provided to carry out an analysis manually, a computer program (WARELOCA, a module in LOGWARE) accompanies this case study.
Given a particular combination of plants, plant capacities, customer service constraints, and warehouses, the program optimally assigns demand centers to warehouses and warehouses to plants by means of linear programming.
From the selected list of warehouses, the least expensive will be chosen if more than one choice is available within the prescribed service distance from a demand center.
If a warehouse cannot be found within the service distance, the warehouse closest to the demand center will be selected.
Only linear variable costs are used in the allocation of demand centers to warehouses.
Storage and capital costs, which are nonlinear, are not used in the allocation process.
They are included in the system costs for a particular configuration.
Fixed costs are neither included in the allocation, nor are they shown in the total system costs.
They must be externally added to system costs.
WARELOCA is a program in which you provide the plant locations and capacities, warehouse locations, customer service distance, and demand and cost levels.
Each run of the program represents an evaluation of a particular network configuration.
The results of a sample WARELOCA run in which the current network is approximated (not the true benchmark) where the existing 4 plants and 22 warehouses are evaluated are given in \color{red}Figure 3\color{black}.

% TODO: incluir fotos da Figura 3, que provavelmente será inútil para o exercício

\section{Solução}

\subsection{1st Question}


The first question is to determine the optimal number of plants and warehouses, both for the current and future scenarios.
To address this, we formulate a Mixed-Integer Linear Programming (MILP) model, which is a classical facility location problem with additional constraints and cost structures.

Conjuntos:

\begin{itemize}
    \item $\mathcal{P}$: Conjunto de plantas
    \item $\mathcal{W}$: Conjunto de armazéns
    \item $\mathcal{D}$: Conjunto de nós de demanda
\end{itemize}

Indices:

\begin{itemize}
    \item $p$: índice para plantas, onde $p \in \mathcal{P}$
    \item $w$: índice para armazéns, onde $w \in \mathcal{W}$
    \item $d$: índice para nós de demanda, onde $d \in \mathcal{D}$
\end{itemize}

% TODO: e se tiver que fechar armazéns atuais?

Variáveis de decisão:

\begin{itemize}
    % Binárias
    \item $z_{w}$: 1 se o armazém $w$ será utilizado e 0 caso contrário
    % Contínuas
    \item $x_{pw}$: quantidade de produto que será enviada da planta $p$ para o armazém $w$
    \item $y_{wd}$: quantidade de produto que será enviada do armazém $w$ para o nó de demanda $d$
    \item $w_{pd}$: quantidade de produto que será enviada diretamente da planta $p$ para o nó de demanda $d$ (envio direto da fábrica para o cliente)
\end{itemize}

Variáveis auxiliares:
\begin{itemize}
    \item $u_{pd}$: se a quantidade de produto que será enviada diretamente da planta $p$ para o nó de demanda $d$ está entre 0 e 100
\end{itemize}

Parâmetros:

\begin{itemize}
    \item $|P|$: número de plantas (para o nosso caso são atualmente 4 e possivelmente 6)
    \item $|W|$: número de armazéns (para o nosso caso são 48)
    \item $|D|$: número de nós de demanda (para o nosso caso são 48)
    \item $S_d$: demanda (\textit{sales}) total do nó de demanda $d$
    \item $c^{in}_{pw}$: custo de transporte da planta $p$ para o armazém $w$
    \item $c^{out}_{wd}$: custo de transporte do armazém $w$ para o nó de demanda $d$
    \item $c^{out'}_{pd}$: custo de transporte da planta $p$ para o nó de demanda $d$
    \item $\rho_{p}$: Custo variável de produção de cada planta $p$
    \item $C_{p}$: Capacidade de produção da planta $p$
    \item $C'_{p}$: Capacidade de estocagem da planta $p$ (para as cargas fracionadas que vão direto para o destino)
    \item $\tau_{w}$: Custo de estocagem do armazém $w$ (tabela 3)
    \item $\epsilon_{w}$: Custo de handling do armazém $w$ (tabela 3)
    \item $\gamma_{w}$: Stock Order Processing (\$/order) (tabela 3)
    \item $\delta_{w}$: Stock Order Size do armazém $w$ (cwt/order) (tabela 3)
    \item $\phi_{w}$: Customer Order Processing (\$/order) (tabela 3)
    \item $\omega_{w}$: Customer Order size (cwt/order) (tabela 3)
    \item $\Gamma$: Valor de uma cwt do produto, medido em \$ (dividimos a venda total pela demanda total, obtendo 108.33)
\end{itemize}

Função objetivo:

% Custo de transporte das plantas para os armazéns
% Custo de transporte dos armazéns para os clientes
% Custo de transporte das fábricas para os clientes
% TODO: Custo de manutenção do estoque (passa a ser um custo fixo...)
% Custo variável de produção

\begin{align}
    \text{Min} \quad
    & \sum_{p \in \mathcal{P}} \sum_{w \in \mathcal{W}} c^{in}_{pw} \cdot x_{pw} \quad +
    && \text{(das fábricas para os armazéns)} \notag \\
    & \sum_{w \in \mathcal{W}} \sum_{d \in \mathcal{D}} c^{out}_{wd} \cdot y_{wd} \quad +
    && \text{(dos armazéns para os clientes)} \notag \\
    & \sum_{p \in \mathcal{P}} \sum_{d \in \mathcal{D}} c^{out'}_{pd} \cdot w_{pd} \quad +
    && \text{(das fábricas para os clientes)} \notag \\
    & \sum_{p \in \mathcal{P}} \rho_{p} \left(
        \sum_{p \in \mathcal{P}} x_{pw} + \sum_{d \in \mathcal{D}} w_{pd}
    \right) \quad +
    && \text{(custo variável de produção)} \notag \\
    & \sum_{p \in \mathcal{P}} \sum_{w \in \mathcal{W}} x_{pw} \cdot \Gamma \cdot \tau_{w} \cdot z_{w} \quad +
    && \text{(custo de estocagem nos armazéns)} \notag \\
    & \sum_{p \in \mathcal{P}} \sum_{w \in \mathcal{W}} 2 \cdot z_{w} \cdot \epsilon_{w} \cdot x_{pw} \quad +
    && \text{(custo de handling nos armazéns)} \notag \\
    % Stock Order costs
    & \sum_{p \in \mathcal{P}} \sum_{w \in \mathcal{W}} \left(
        x_{pw} \cdot \frac{\gamma_{w}}{\delta_{w}} \cdot z_{w}
        \right) \quad +
    && \text{(custo de processamento do pedido de estoque)} \notag \\
    % Customer Order costs
    & \sum_{p \in \mathcal{P}} \sum_{w \in \mathcal{W}} \left(
        y_{wd} \cdot \frac{\phi_{w}}{\omega_{w}} \cdot z_{w}
        \right)
    && \text{(custo de processamento do pedido do cliente)} \notag \\
\end{align}

Sujeito a: % (Restrições)

\begin{equation} % Demanda total deve ser atendida
    \sum_{w \in \mathcal{W}} y_{wd} +
    \sum_{p \in \mathcal{P}} w_{pd} =
    S_d, \quad
    \forall d \in \mathcal{D}
\end{equation}


\begin{equation} % Capacidade das plantas (obs.: talvez multiplicar por 0.8 pra ter uma folga)
    \sum_{w \in \mathcal{W}} x_{pw} +
    \sum_{d \in \mathcal{D}} w_{pd}
    \leq C_p, \quad
    \forall p \in \mathcal{P}
\end{equation}

\begin{equation} % Cada armazém aberto precisa ter no mínimo 10400 cwt para ser viável
    \sum_{d \in \mathcal{D}} y_{wd}
    \geq 10400 \cdot z_{w}, \quad
    \forall w \in \mathcal{W}
\end{equation}

% TODO: cuidado com conversão entre libras e cwt

\begin{equation} % limite de estoque de cada planta (para as cargas fracionadas que vão direto para o destino)
    \sum_{d \in \mathcal{D}} u_{pd} \cdot w_{pd} \leq C'_p \quad \forall p \in \mathcal{P}
\end{equation}

\begin{equation}% Expressão para u_{pd}:
    u_{pd} =
    \begin{cases}
        1, & \text{se } 0 \leq w_{pd} \leq 100 \\
        0, & \text{caso contrário}
    \end{cases}, \quad \forall p \in \mathcal{P}, \forall d \in \mathcal{D}
\end{equation}


\begin{equation} % Balanço de massa do armazém
    \sum_{p \in \mathcal{P}} x_{pw} =
    \sum_{d \in \mathcal{D}} y_{wd}, \quad
    \forall w \in \mathcal{W}
\end{equation}

Espaço das variáveis:

\begin{itemize}
    \item $x_{pw} \in \mathbb{R}^+$
    \item $y_{wd} \in \mathbb{R}^+$
    \item $w_{pd} \in \mathbb{R}^+$
    \item $z_{w} \in \{0, 1\}$
    \item $u_{pd} \in \{0, 1\}$
\end{itemize}


% (vamos deixar de fora) limite de distancia de atendimento (no máximo 300 milhas)


\subsection{2nd Question}
\cite{states_centers}
\cite{ballou}

\subsection{3rd Question}


\subsection{4th Question}


\subsection{5th Question}


\subsection{6th Question}

\bibliographystyle{abntex2-alf}
\bibliography{refs}


\end{document}

